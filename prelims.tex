\chapter{Preliminaries}\label{pre} %\footnote{See Section 0 in \cite{F}.}

\section{Notation}
To start, we begin learning and/or reviewing some notation.  Here is a table of frequently used symbols and abbreviations we will use, along with their meanings.
\begin{center}
\begin{threeparttable}
\renewcommand{\arraystretch}{1.3}
\begin{tabular}{|l|l|}
\hline
\textbf{Notation}& \textbf{Meaning}\\
\hline
$\forall$ & for all/for every\\
\hline
$\exists$ & there exists\\
\hline
s.t. & such that\tnote{*}\\
\hline
$\therefore$ & therefore\\
\hline
WLOG or wolog & without loss of generality\tnote{$\dagger$}\\
\hline
WTS & want to show\\
\hline
! & unique (also used to denote factorials)\\
\hline
$\Rightarrow$ & implies; also the ``if" direction in proofs\\
\hline
$\Leftarrow$ & only if; also the ``only if" direction in proofs \\
\hline
$\Leftrightarrow$ or iff& if and only if\\
\hline
w/ & with\\
\hline
o.w. & otherwise\\
\hline
\end{tabular}
\begin{tablenotes}
\item[*]\small You can also use $\ni$ to denote ``such that," but we will avoid this convention as it can be confused with the notation $\in$, meaning ``is an element of."
\item[$\dagger$]\small We will discuss what this means in greater detail as it arises in our work.
\end{tablenotes}
\end{threeparttable}
\end{center}

Finally, the notation $:=$ means we are assigning a definition. E.g., writing ``Let $S:=\{1,2,3\}$" means we are defining $S$ to be the set $\{1,2,3\}$.  We often omit the colon and simply write $=$ in these cases, but we may include it to emphasize the fact that we are defining something.

\section{Sets}
Now we provide a ``definition" of a basic mathematical object to which we will soon add bells and whistles.

\begin{df}{``Sort of" definition} A \textit{set} is an (unordered)
collection of objects.\end{df}

We say this is a ``sort of" definition because it is not a
rigorous definition of a set.  For instance, what do we mean by a
``collection" of objects? This ``definition" will be sufficient for
our course, but be warned that defining a set in this vague way can
lead to some serious mathematical issues, such as Russell's
paradox\footnote{ Let $S$ be the set of all sets that aren't members
of themselves.  Is $S$ a member of itself? If you think carefully
about this, you'll see that $S$ can be neither a member of itself,
nor \underline{not} a member of itself. Uh oh!  This contradiction
is known as Russell's paradox (named for the British philosopher, mathematician, and all-round academic Bertrand Russell). Mathematicians deal with this by
declaring that some object collections, called \textit{classes}, are
not in fact sets.};  a mathematician whose expertise is in set
theory may scowl disagreeably if you try to define a set as we have
above.

\begin{df}{Definitions and notation} The members of a set are called its \textit{elements}. If
$S$ is a set, we write $x\in S$ to indicate ``$x$ is an element
of $S$," and $x \not\in S$ to indicate ``$x$ is not an element
of $S$."  There is a unique set containing no elements; it is
called the \textit{empty set}, and denoted by $\emptyset$.\end{df}


Sets must be \textit{well defined}: that is, it must be clear
exactly which objects are in a set, and which objects aren't.
For instance, the set of all integers is well defined, but the
set of all big integers is not well defined, since it is
unclear what ``big" means in this context.

We refer to some sets so frequently in mathematics that we have special notation for them.  Common examples are:


\begin{center}
\renewcommand{\arraystretch}{1.3}
\begin{threeparttable}
\begin{tabular}{|l|l|}
\hline
\textbf{Notation}& \textbf{Meaning}\\
\hline
$\Z$ & the set of all integers\\
\hline
$\fatq$ & the set of all rational numbers\\
\hline
$\fatr$ & the set of all real numbers\\
\hline
$\fatc$ & the set of all complex numbers\\
\hline
$\fatn$ & the set of all natural numbers (that is, the set $\{0, 1, 2, \ldots\}$)\tnote{*}\\
\hline
$\Z^+$/$\fatq^+$/$\fatr^+$ & the set of all positive integers/rational numbers/real numbers\\
\hline
$\Z^-$/$\fatq^-$/$\fatr^-$ & the set of all negative integers/rational numbers/real numbers\\
\hline
$\Z^-$/$\fatq^-$/$\fatr^-$/$\fatc^*$ &   \pbox{20cm}{the set of all nonzero integers/rational numbers/real numbers/\\complex numbers}\\
\hline
\end{tabular}
\begin{tablenotes}
\item[*]\small Be aware that many books/mathematicians do not include 0 in the set of natural numbers.
\end{tablenotes}
\end{threeparttable}
\end{center}

We also provide notation for commonly considered sets of matrices:

\begin{df}{Notation} Given $m,n\in \Z^+$ and a set $S$, we define $\M_{m\times n}(S)$ to
be the set of all $m\times n$ matrices over $S$ (that is, of all $m\times n$ matrices with entries in $S$).  We use the
shorthand notation $\M_n(S)$ for the set $\M_{n\times n}(S)$.\end{df}

One common way of describing a set is to list its elements in
curly braces, separated by commas; you can use ellipses to
indicate a repeated pattern of elements. A few examples are
$\{1,4,\pi\}$, $\{3, 4, 5, \ldots\}$, and $\{\ldots, -4, -2, 0,
2, 4, \ldots\}$; the last of these can be written more concisely as $\{0,\pm 2, \pm 4,\ldots\}$.  Note that since elements of a set are
unordered, the sets $\{1,4,\pi\}$ and $\{4,\pi, 1\}$, for
instance, are identical.

 Another method is using \textit{set-builder notation}. This consists of an element name (or
names), followed by a colon (meaning ``such that"), followed by
a Boolean expression involving the element name(s), all
surrounded by curly braces.\footnote{
 \textbf{WARNING:} The use of a colon to denote ``such that" is \textit{only} valid in the above set-builder notation context.  Outside of this context, you should never use a colon to denote ``such that"; instead, write out the actual words or use the abbreviation s.t. (or the symbol $\ni$, though I do not recommended).  Conversely, never use one of those ways of indicating ``such that" within set-builder notation; always use a colon there.  Why?  Convention.}
 For example,
$$\{x\in \Z : x > 4\}$$ is the set $\{5, 6, 7, \ldots\}$, while
$$\{z\in \fatc : |z|=1\}$$ is the set of all complex numbers at
distance 1 from the origin in the complex plane.

\begin{df}{Note} If one simply writes $\{x\,:\,x>4\}$,
it is unclear what this set is; it could be the set of all
integers greater than 4, or the set of all real numbers greater
than 4, or something else. When one can, it is better to
identify the named element(s) as a member (members) of a known
set, such as $\fatr$ or $\Z$, whenever possible.\end{df}

\begin{df}{Definitions and notation} Set $A$ is a \textit{subset} of $B$ (and set $B$ is a \textit{superset} of $A$) if every element in $A$ is also in $B$. We
denote ``$A$ is a subset of $B$'' by $A\subseteq B$. Sets $A$
and $B$ are said to be \textit{equal}, and we write $A=B$, if they
contain exactly the same elements; equivalently, $A=B$ if and
only if $A \subseteq B$ and $B\subseteq A$.  Set $A$ is a \textit{proper} subset of set $B$ if $A\subseteq B$ but $A\neq B$; we
write this $A\subsetneq B$ or $A\subset B$ \footnote{ \textbf{
WARNING:} Sometimes the notation $A\subset B$ is merely used to
indicate that $A$ is a subset of $B$ (but may equal $B$). Be
careful to check what your book or peer means by this
notation.}. \end{df}

\begin{df}{Remark} One often proves that two sets $A$ and
$B$ are equal by proving that $A\cont B$ and $B\cont A$.\end{df}

\begin{example}{} We have the following: $\Z^+ \subseteq \Z \subseteq \fatq \subseteq \fatr \subseteq \fatc$. \end{example}


\begin{df}{Definition} The \textit{power set} of $A$, denoted $P(A)$, is the set of
all subsets of $A$. (Note that the power set of any set
contains the empty set as an element.)\end{df}

\warn{Be careful to use your curly braces correctly when writing power sets!
Remember, the power set of a set is a \textit{set of sets}.}

The following provides a good example of using braces correctly.

\begin{example}{} If $A=\{a,b\}$, then $P(A)=\{\emptyset, \{a\}, \{b\}, \{a,b\}\}$.  Note that the element $\{a,b\}$ of $P(A)$ could also be written simply as $A$. \end{example}

\begin{df}{Definitions and notation}\

\begin{enumerate}
\item If $A$ and $B$ are sets, then the \textit{union} of $A$ and $B$, denoted $A\cup B$, is the set
$A\cup B=\{x: x\in A \mbox{ or } x\in B\};$ the \textit{intersection} of $A$ and $B$, denoted $A\cap B$, is the set
$A\cap B=\{x: x\in A \mbox{ and } x\in B\};$ and the \textit{difference} of $A$ and $B$, denoted $A-B$ (or $A\setminus B$), is the set
$A-B=\{x: x\in A \mbox{ and } x\not\in B\}.$

\item More generally, given any collection of sets $A_i$ ($i$ in some index set $I$), the \textit{union} of the $A_i$ is
$$\bigcup_{i\in I}A_i=\{x: x\in A_i \mbox{ for some } i\in I\},$$ and the \textit{intersection} of the $A_i$ is
$$\bigcap_{i\in I}A_i=\{x: x\in A_i \mbox{ for every } i\in I\}.$$

\item Sets $A$ and $B$ are \textit{disjoint} if $A\cap B=\emptyset$.  More generally, sets $A_i$ ($i$ in some index set $I$) are
\textit{disjoint} if $$\bigcap_{i\in I}A_i=\emptyset$$ and are {\it mutually disjoint} if $$A_i\cap A_j=\emptyset \mbox{ for all }i\neq j \in I.$$
\end{enumerate}
\end{df}

Notice that for any sets $A$ and $B$, $A\cap B \cont A \cont A\cup B$ and
$A\cap B \cont B \cont A\cup B$. Also note that if sets $A_i$ ($i \in I$) are mutually disjoint then they are also disjoint, but they may be disjoint without being mutually disjoint. For example, the sets $\{i, i+1\}$ for $i\in \Z$ are disjoint but not mutually disjoint. (Do you see why?)

We define one more way of ``combining" sets.

\begin{df}{Definitions} Let $A$ and $B$ be sets.  Then the \textit{(direct)
product} $A\times B$ of $A$ and $B$ is the set $$A\times B =
\{(a,b): \mbox{$a\in A$, $b\in B$}\}.$$  An element $(a,b)$ of
$A\times B$ is called an \textit{ordered pair}. More generally, if
$A_1, A_2, \ldots, A_n$ are sets for some $n\in \Z^+$, then the
\textit{product} of the $A_i$ is $$A_1\times A_2 \times \cdots
\times A_n=\{(a_1, a_2, \ldots, a_n): a_i \in A_i \mbox{ for
each }i\};$$  the elements $(a_1,a_2,\ldots,a_n)$ of this
product are called $n$-tuples (or triples, if $n=3$).\footnote{
You can also have products of infinitely many sets, but we will
not discuss that in this course.} Finally, if each set $A_i$ is
the same set $A$, we can use the notation $A^n$ to denote the
product $$A\times A \times \cdots \times A$$ of $n$ copies of
$A$.\end{df}

\begin{example}{} For example, the Cartesian plane is the set $\fatr^2$, and the set $\Z \times \fatr$ consists of exactly the points in the plane with integer $x$-coordinates
(that is, the points that lie on vertical lines intersecting the $x$-axis at integer values). \end{example}

\section{Functions}

You have probably encountered functions before.  In introductory
calculus, for instance, you typically deal with functions from
$\fatr$ to $\fatr$ (e.g., the function $f(x)=x^2$).  More generally,
functions ``send" elements of one set to elements of another set;
these sets may or may not be sets of real numbers. We provide below
a ``good enough for government work" definition of a function.
%\textcolor[rgb]{1.00,0.00,0.00}{We leave the ``official" definition of a function for later; for now, the following definition will be sufficient.}

\begin{df}{Definitions and notation} A \textit{function} $f$ from a set $S$ to a set $T$ is a
``rule" that assigns to each element $s$ in $S$ a unique
element $f(s)$ in $T$; the element $f(s)$ is called the \textit{image of $s$ under $f$}.  If $f$ is a function from $S$ to $T$,
we write $f: S \to T$, and call $S$ the \textit{domain} of $f$ and
$T$ the \textit{codomain} of $f$. The \textit{range} of $f$ is
$$f(S)=\{f(s) \in T : s \in S\} \cont T.$$ More generally, if $U \cont S$, the \textit{image of $U$ in $T$ under $f$} is
$$f(U)=\{f(u)\in T : u\in U\}.$$ If $V\cont T$, the \textit{preimage of $V$ in $S$ under $f$} is the set
$$f^{\leftarrow}(V)=\{s\in S: f(s)\in V\}.$$
\end{df}

\begin{example}{} Consider the function $f: \Z \to \fatr$ defined by $f(x)=x^2$.
The domain of $f$ is $\Z$ and the codomain of $f$ is $\fatr$; the
range of $f$ is $\{x^2\,:\,x\in \Z\}=\{0,1,4,9,\ldots\}$. The image
of $\{-2,-1,1,2\}$ under $f$ is the two-element set $\{1,4\} \cont
\fatr$, and the preimage of $\{4,25,\pi\}$ under $f$ is the set
$\{\pm 2, \pm 5\}$. (Do you see why $\pm \sqrt{\pi}$ are not in this
preimage?) What is the preimage of just $\{\pi\}$ under $f$? \end{example}


The following definitions will be very important in our future work.


\begin{df}{Definitions} Let $S$ and $T$ be sets, and $f:S\to T$.
\begin{enumerate}
\item Function $f$ is \textit{one-to-one} (1-1) if whenever $s_1, s_2\in S$ with $f(s_1)=f(s_2)$, we have $s_1=s_2$.  Equivalently, $f$ is one-to-one if
whenever $s_1\neq s_2 \in S$, then $f(s_1)\neq f(s_2) \in T$.
\item Function $f$ is \textit{onto} if for every $t\in T$, there exists an element $s\in S$ such that $f(s)=t$.  Equivalently, $f$ is onto if $f(S)=T$.
\item Function $f$ is a \textit{bijection} if it is both one-to-one and onto.
\end{enumerate}
\end{df}

We will often have to show functions are one-to-one or onto.  Given a function $f:S\to T$, the following methods are recommended.

\begin{center}
\renewcommand{\arraystretch}{1.3}
\begin{tabular}{|l|p{5cm}|}
\hline
\textbf{To prove that $f$ is one-to-one}& Let $s_1,s_2 \in S$ with $f(s_1)=f(s_2)$ and prove that then $s_1=s_2$.  {\bf WARNING:} It is \underline{not} sufficient to prove that if $s_1=s_2$ in $S$ then $f(s_1)=f(s_2)$; that is true for any function from $S$ to $T$!  Be careful to \textit{assume} and \textit{prove} the correct facts.\\
\hline
\textbf{To prove that $f$ is \textit{not} one-to-one}& Identify two elements $s_1 \neq s_2$ of $S$ such that $f(s_1)=f(s_2)$.\\
\hline
\textbf{To prove that $f$ is onto}& Let $t\in T$ and prove that there exists an element $s\in S$ with $f(s)=t$. {\bf WARNING:} It is \underline{not} sufficient to prove that if $s\in S$ then $f(s)$ is in $T$; that is true for any function from $S$ to $T$!  Again, be careful to \textit{assume} and \textit{prove} the correct facts.\\
\hline
\textbf{To prove that $f$ is \textit{not} onto}& Identify an element $t\in T$ for which there is no $s\in S$ with $f(s)=t$.\\
\hline
\end{tabular}
\end{center}
\smallskip

\begin{example}{} Consider the function $f: \fatr^* \to \fatr^+$ defined by $f(x)=x^2$. Function $f$ is \textit{not} one-to-one: indeed,
$-1$ and $1$ are in $\fatr^*$ with $f(-1)=1=f(1)$ in $\fatr^+$.
However, $f$ \textit{is} onto: indeed, let $t\in \fatr^+$.  Then
$\sqrt{t} \in \fatr^*$ with $t=f(\sqrt{t})$, so we're done. \end{example}

\begin{example}{} Consider the function $f: \Z^+ \to \fatr$ defined by $f(x)=x/2$. Function $f$ \textit{is} one-to-one: indeed, let $s_1, s_2 \in \Z^+$ with $f(s_1)=f(s_2)$.
Then $s_1/2=f(s_1)=f(s_2)=s_2/2$; multiplying both sides of the equation $s_1/2=s_2/2$ by 2, we obtain $s_1=s_2$. However, $f$ is \textit{not} onto: for example, $\pi\in \fatr$
but there is no positive integer $s$ for which $f(s)=s/2=\pi$.\end{example}


Recall that we can combine certain functions using \textit{composition}:
if $f:S\to T$ and $g:T\to U$, then \textit{$g$ composed with $f$} is the function $g\circ f: S\to U$ defined by
$$(g\circ f)(s)=g(f(s))$$ for all $s\in S$.\footnote{More generally, you can compose functions $f:S\to T$ and $g:R\to U$ to form $g\circ f:S\to U$,  as long as $f(S)\subseteq R$.} Also recall that given any set $S$, the \textit{identity function on $S$} is the function $1_S: S\to S$ defined by $1_S(s)=s$ for every $s\in S$.


\begin{df}{Definitions} Let $f$ be a function from $S$ to $T$.  A function $g$
from $T$ to $S$ is an \textit{inverse} of $f$ if $g\circ f$ and
$f\circ g$ are the identity functions on $S$ and $T$,
respectively; that is, if for all $s\in S$ and $t\in
T$, $g(f(s))=s$ and $f(g(t))=t$.  If $f$ has an inverse, we say that $f$ is \textit{invertible}.\end{df}

\smallskip
We present three theorems, omitting the
proofs of the first two.
\begin{thm}\label{}
If $f$ has an inverse, then that inverse is unique. \end{thm}

\begin{df}{Notation} If $f$ is invertible, we denote its unique inverse by $f^{-1}$.
\end{df}

\begin{thm}\label{invbij}
Function $f:S\to T$ has an inverse if and only if $f$ is a bijection. Also, if $f$ has inverse $f^{-1}$, then $f^{-1}$ is also a bijection. \end{thm}

\begin{thm}\label{compbij} Let $f:S\to T$ and $g:T\to U$ be functions.
If $f$ and $g$ are both 1-1 [onto], then so is $g\circ f: S\to U$.
\end{thm}

\begin{proof}
Assume $f$ and $g$ are 1-1.  Let $s_1, s_2\in S$
with $(g\circ f)(s_1)=(g\circ f)(s_2)$.  Then $g(f(s_1))=g(f(s_2))$;
since $g$ is one-to-one (since it's a bijection), this shows that
$f(s_1)=f(s_2)$. Then since $f$ is one-to-one (since $f$ is also a
bijection), we must have $s_1=s_2$. Thus, $g\circ f$ is one-to-one.\end{proof}

\red{The proof that $g\circ f$ is onto if $f$ and $g$ are onto is left as an exercise for the reader.}

\section{Cardinality}

One of the set traits that will be useful to us in
distinguishing between algebraic structures is \textit{cardinality}.

\begin{df}{Definitions} A set is \textit{finite} if it contains a finite number of
elements (including the case in which it contains no elements);
otherwise, it's \textit{infinite}.\end{df}

\begin{df}{Definition}The \textit{cardinality of a finite set} is the number of
elements in that set. \end{df}

But what is the cardinality of an infinite set?  This is more
subtle. Essentially, it is the ``size" of the set, but what
does that mean?

\begin{df}{Definitions and notation} The symbol $\aleph_0$ denotes the cardinality of the set
$\Z$.  An arbitrary set $S$ has cardinality $\aleph_0$ if there
exists a bijection from $S$ to $\Z$ (equivalently, if there
exists a bijection from $\Z$ to $S$).

Note that if there is a bijection from $S$ to $\Z$, then $S$
must be infinite.  If the cardinality of $S$ is $\aleph_0$, we
say $S$ is \textit{countably infinite}. If $S$ is finite or
countably infinite, we say it's \textit{countable}.  Finally, an
infinite set is \textit{uncountably infinite}, or \textit{uncountable}, if it is not countably infinite.  In this class,
we don't discuss the cardinality of uncountably infinite sets,
but simply note that an uncountable set cannot have the same
cardinality as a countable one.\end{df}

\begin{df}{Notation} We denote the cardinality of a set $S$ by
$|S|$.\footnote{ Of course, vertical bars are used to denote
other mathematical concepts; for instance, if $x$ is a real
number, $|x|$ usually denotes the absolute value of $x$. You
must determine from context, and from the nature of the
expression within the bars, what vertical bars mean in a
particular situation.}\end{df} So, e.g., $|\{a,b\}|=2$.

\begin{example}{}  Clearly, $\Z$ itself is countably
infinite.\end{example}

Perhaps surprisingly, a proper subset of a set can have the
same cardinality as its superset, as the following example
shows.

\begin{example}{} We claim that $\Z^+$ is countably infinite. Indeed, consider the function $f:\Z^+ \to \Z$ defined by $f(n)=(-1)^n \lfloor n/2 \rfloor$, where $\lfloor x \rfloor$
denotes the greatest integer less than or equal to $x$, for each $x\in \fatr$. The fact that $f$ is a bijection is demonstrated (though not proven) by the following visual representation,
where each number maps via $f$ to the value directly below it:

$$
\begin{array}{llrrrrrrr}
&\Z^+ & 1 & 2 & 3 & 4 & 5 & \ldots&\\
f &\Big\downarrow&&&&&&&\\
&\Z   & 0 & 1 & -1 & 2 & -2 & \ldots&\\
\end{array}$$

Note that this means that $\Z$ and its proper subset $\Z^+$
have the same cardinality, that is, the same ``size"! \end{example}

 We summarize here examples of countably and uncountably
infinite sets. (On pp. 5--6 of \cite{F}, Fraleigh sketches proofs of the
facts that $\fatq$ is countable and that the interval $(0,1)$
in $\fatr$ is uncountable. The proof then that $\fatr$ is
uncountable follows from Theorem \ref{cardthm}, which
follows.)

\begin{center}
\renewcommand{\arraystretch}{1.3}
\begin{tabular}{|l|l|}
\hline
Countably infinite sets: &$\Z$, $\Z^+$, $\Z^-$, $\Z^*$, $\fatq$, $\fatq^+$, $\fatq^-$, $\fatq^*$, $\fatn$\\
\hline
Uncountably infinite sets: &$\fatr$, $\fatr^+$, $\fatr^-$, $\fatr^*$, $\fatc$, $\fatc^*$,  the interval $(0,1)$ in $\fatr$                              \\
\hline
\end{tabular}
\end{center}

 The key idea here for us is that if two sets are
essentially ``the same," then they must have the same ``size."
Thus, we see that there is some fundamental difference between
the sets $\Z$ and $\fatr$ (in fact, there are many such
differences). On the other hand, cardinality alone won't allow
us to distinguish structurally between the sets $\Z$ and
$\Z^+$.

We end our preliminary chapter with the following theorem and a corollary of it (which can be proved using induction on $n$).

\begin{thm}\label{cardthm} Let $A$ and $B$ be sets.

\begin{enumerate}
\item If $A\cont B$ and $A$ is infinite [uncountable] then so is $B$.

\item If $A\cont B$ and $B$ is finite [countable] then so is $A$.

\item If $|A|=n<\infty$ and $|B|=m< \infty$, then $|A\times B|=mn$.

\item Any finite product of countable sets is countable.\footnote{\textbf{Note.} It is \textbf{not} true that any countable product
of countable (or even finite) sets is countable. Indeed, even
the set $\{0,1\}\times \{0,1\}\times \cdots$ is
uncountable. (If you want to get into the gory details
of this, the key is that there is a bijection from this set to
the power set of the natural numbers, which Cantor's Theorem
tells us is uncountable.  You are welcome to jump down this
rabbit hole by googling ``Cantor's Theorem," if you desire, but
know that you will not be responsible for that material in
class.)}

%\qquad Any countable union of countable sets is countable.
\end{enumerate}

\end{thm}

\begin{proof} We omit proofs of the statements in Parts 1 and 2. \red{The proof of the statement in Part 3 is left as an exercise for the reader.}\end{proof}

For Part 4: Let $A$ and $B$ be countable sets.  Assume that $A$ and $B$ are both countably infinite. Since $\Z^+$ is countably infinite, we can index the elements of $A$ and of $B$ by $\Z^+$, writing $$A=\{a_1,a_2,\ldots\} \mbox{\quad and \quad} B=\{b_1,b_2,\ldots\}.$$ Notice that the table

$$
    \begin{array}{cccc}
      (a_1,b_1) & (a_1,b_2) & (a_2,b_3) & \cdots \\
      (a_2,b_1) & (a_2,b_2) & (a_2,b_3) & \cdots \\
      (a_3,b_1) & (a_3,b_2) & (a_3,b_3) & \cdots\\
      \vdots & \vdots & \vdots &  \ddots
    \end{array}$$ contains every element of $A\times B$. We can then list the elements of $A\times B$ by listing the elements in progressive upper-right to lower-left diagonals, starting with $(a_1,b_1)$ and moving to the right along the top row: that is, we can write
   $$A\times B=\{(a_1,b_1),(a_1,b_2),(a_2,b_1),(a_2,b_3),(a_2,b_2),(a_3,b_1),\ldots\}$$
    This implicitly yields a bijection from $\Z^+$ to $A\times B$; thus, $A\times B$ is countably infinite, and hence countable.

    The proof in the case that one or both of sets $A$ and $B$ are finite is similar; the corresponding table in that case will simply have either only finitely many rows or finitely many columns, or both.

\begin{cor} Let $n>1$ be an integer and let $A_1,A_2,\ldots, A_n$ be countable sets.  Then $A_1\times A_2\times \cdots \times A_n$ is countable.
\end{cor}


\pagebreak

\section{Exercises}

\begin{exercise}[ID=1A]

Yes/No. For each of the following, write Y if the object described is a well-defined set; otherwise, write N. You do NOT need to provide explanations or show work for this problem.

\begin{enumerate}

\item $\{z \in \fatc \,:\, |z|=1\}$

\item $\{\epsilon \in \fatr^+\,:\, \epsilon \mbox{ is sufficiently small}\}$

\item $\{q\in \fatq \,:\, q \mbox{ can be written  with denominator }4\}$

\item $\{n \in \Z\,:\, n^2 <0\}$

\end{enumerate}

\end{exercise}
\begin{solution}[print=true]
\begin{inparaenum}[(a)]
\item Y \hfill \item N \hfill \item  Y \hfill \item Y (it's the empty set)
\end{inparaenum}
\end{solution}

\begin{exercise}[ID=1B]
List the elements in the following sets, writing your answers as sets.

\medskip
\textbf{Example:} $\{z\in \fatc\,:\,z^4=1\}$  \quad \textbf{Solution:} $\{\pm 1, \pm i\}$
\medskip

\begin{enumerate}

\item $\{z\in \fatr\,:\, z^2=5\}$

\item $\{m \in \Z\,:\, mn=50 \mbox{ for some }n\in \Z\}$

\item $\{a,b,c\}\times \{1,d\}$

\item $P(\{a,b,c\})$
\end{enumerate}

\end{exercise}

\begin{solution}[print=true]
\begin{enumerate}
\item $\{\pm\sqrt{5}\}$
\item $\{\pm 50, \pm 25, \pm 10, \pm 5, \pm 2, \pm 1\}$
\item $\{(a,1),(a,d), (b,1),(b,d),(c,1),(c,d)\}$
\item $\{\emptyset, \{a\}, \{b\},
    \{c\},\{a,b\},\{a,c\},\{b,c\}, \{a,b,c\}\}$
\end{enumerate}
\end{solution}

\begin{exercise}[ID=1C]
Let $S$ be a set with cardinality $n\in \fatn$. Use the cardinalities of $P(\{a,b\})$ and $P(\{a,b,c\})$ to make a conjecture about the cardinality of $P(S)$. You do not need to prove that your conjecture is correct (but you should try to ensure it is correct).

\end{exercise}

\begin{solution}[print=true]
$|P(\{a,b\})|=4=2^2$ and $|P(\{a,b,c\})|=8=2^3$; we may conjecture that when $|S|=n$, $|P(S)|=2^n$.

\end{solution}

\begin{exercise}[ID=1D]
Let $f: \Z^2 \to \fatr$ be defined by $f(a,b)=ab$.  (Note: technically, we should write $f((a,b))$, not $f(a,b)$, since $f$ is being applied to an ordered pair, but this is one of those cases in which mathematicians abuse notation in the interest of concision.)

\begin{enumerate}
\item What are $f$'s domain, codomain, and range?
\item Prove or disprove each of the following statements. (Your proofs do not need to be long to be correct!)
\begin{enumerate}
\item $f$ is onto;
\item $f$ is 1-1;
\item $f$ is a bijection. (You may refer to parts (i) and (ii) for this part.)
\end{enumerate}

\item Find the images of the element $(6,-2)$ and of the set $\Z^- \times \Z^-$ under $f$. (Remember that the image of an element is an element, and the image of a set is a set.)
\item Find the preimage of $\{2,3\}$ under $f$. (Remember that the preimage of a set is a set.)
\end{enumerate}

\end{exercise}

\begin{solution}[print=true]

\begin{enumerate}
\item $f$'s domain, codomain, and range are, respectively, $\Z^2$, $\fatr$, and $\Z$.

\item
\begin{enumerate}
\item $f$ is not onto, since, for instance, $1/2\in \fatr-\Z$.
\item $f$ is not 1-1: for instance, $f(-2,2)=-4=f(2,-2)$.
\item $f$ is not a bijection since it's not 1-1. (It would be equally valid to answer that it's not a bijection since it's not onto, or that it's not a bijection since it's neither 1-1 nor onto.)
\end{enumerate}

\item $f(6,-2)=-12$ and $f(\Z^-\times \Z^-)=\Z^+$.

\item
\begin{align*}
f^{\leftarrow}(\{2,3\})&=\{(a,b)\in \Z\times \Z\,:\, f(a,b)\in \{2,3\}\}\\
&=\{(a,b)\in \Z\times \Z\,:\, ab=2 \mbox{ or } ab=3\},
\end{align*}
which is the set
$$\{(1,2),(2,1),(-1,-2),(-2,-1),(1,3),(3,1),(-1,-3),(-3,-1)\}.$$
\end{enumerate}

\end{solution}

\begin{exercise}[ID=1F]
Let $S$, $T$, and $U$ be sets, and let $f: S\to T$ and $g: T\to U$ be onto.  Prove that $g \circ f$ is onto.
\end{exercise}

\begin{solution}[print=true]
Let $u\in U$. We want to show there is an element of $S$ that gets mapped to $u$ by $g\circ f$. Since $g:T\to U$ is onto, there is an element $t\in T$ such that $g(t)=u$; next, since $f:S\to T$ is onto, there is an element $s\in S$ such that $f(s)=t$.  Then $(g\circ f)(s)=g(f(s))=g(t)=u$.  Thus, $g\circ f$ is onto.
\end{solution}

\begin{exercise}Let $|A|=n<\infty$ and $|B|=m< \infty$. Prove that $|A\times B|=mn$.
\end{exercise}

\begin{solution}[print=true]
We can list the elements of $A$ and $B$ as so: $$A=\{a_1,a_2,\ldots, a_n\} \mbox{\quad and \quad}B=\{b_1,b_2,\ldots, b_m\}.$$
Consider the table
$$\begin{array}{cccc}
      (a_1,b_1) & (a_1,b_2) & \cdots & (a_1,b_m) \\
      (a_2,b_1) & (a_2,b_2) & \cdots & (a_2,b_m)\\
      \vdots & \vdots  & \ddots & \vdots\\
      (a_n,b_1) & (a_n,b_2) & \cdots &  (a_n,b_m).
    \end{array}$$
Clearly this table contains $mn$ elements, and contains each element of $A\times B$ exactly once.  Therefore, $|A\times B|=mn$.

\end{solution} 