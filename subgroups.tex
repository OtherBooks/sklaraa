\chapter{Subgroups}\label{subgps}

\section{Introduction to subgroups}

Sometimes groups are too complicated to understand directly.  One method that can be used to identify a group's structure is to study its \textit{subgroups}.

\begin{df}{Definitions and notation} A \textit{subgroup} of a group $G$ is a subset of $G$ that
is also a group under $G$'s operation. If $H$ is a subgroup of
$G$, we write $H \leq G$; if $H\cont G$ is not a subgroup of
$G$, we write $H\not\leq G$.\end{df}

 \warn{Do not confuse sub\textbf{groups} with sub\textbf{sets}!
All subgroups of a group $G$ are, by definition, subsets of
$G$, but not all subsets of $G$ are subgroups of $G$ (see
Example \ref{subsetvsubgp1}, Parts 2--5, below). Whether or not
a subset of $G$ is a subgroup of $G$ depends on the operation
of $G$.}


\begin{example}{subsetvsubgp1}\

\begin{enumerate}
\item
Consider the subset $\Z$ of the group $\fatq$, assuming that $\fatq$ is equipped with the usual addition of real numbers (as we indicated above that we would assume, by default).  Since we already know that $\Z$ is a group under this operation, $\Z$ is not just a subset but in fact a subgroup of $\fatq$ (under addition).

\item
Instead, consider the subset $\fatq^+$ of the group $\fatq$.  This
subset is \underline{not} a group under $\fatq$'s operation $+$,
since it does not contain an identity element for $+$.  Therefore,
$\fatq^+$ is a subset but not a subgroup of $\fatq$.

\item Let $I$ be the subset $$I=\fatr-\fatq=\{x\in \fatr: x
    \mbox{ is irrational}\}$$ of the group $\fatr$. The set
    $I$ is \underline{not} a group under $\fatr$'s
    operation $+$ since it is not closed under addition:
    for instance, $\pi, -\pi \in I$, but
    $\pi+(-\pi)=0\not\in I$.  So $I$ is a subset but not a
    subgroup of $\fatr$.

\item Consider the subset $\Z^+$ of the group $\fatr^+$.
The set $\Z^+$ is closed under multiplication, multiplication is
associative on $\Z^+$, and $\Z^+$ does contain an identity element
(namely, 1).  However, most elements of $\Z^+$ do not have inverses
in $\Z^+$ under multiplication: for instance, the inverse of $3$
would have to be $1/3$, but $1/3\not\in \Z^+$.  Therefore, $\Z^+$ is
a subset but not a subgroup of $\fatr^+$.

\item Consider the subset $GL(n,\fatr)$ of $\M_n(\fatr)$.  We know that $GL(n,\fatr)$ is a group, so it might be tempting to say that it is a subgroup of $\M_n(\fatr)$; to be a subgroup of $\M_n(\fatr)$, $GL(n,\fatr)$ must be a group under $\M_n(\fatr)$'s operation, which is matrix addition.  While $GL(n,\fatr)$ is a group under matrix multiplication, it is not a group under matrix addition: for instance, it is not closed under matrix addition, since $I_n, -I_n\in GL(n,\fatr)$ but $I_n+(-I_n)$ is the matrix consisting of all zeros, which is not in $GL(n,\fatr)$. So $GL(n,\fatr)$ is a subset but not a subgroup of $\M_n(\fatr)$.

\item Consider the subset $H=\{0,2\}$ of $\Z_4$.  The subset $H$ is closed under addition modulo 4 ($0+0=0$, $0+2=2+0=2$, $2+2=0$), addition modulo 4 is always associative, $H$ contains an identity element (namely, 0) under addition modulo 4, and both 0 and 2 have inverses in $\Z_4$ under this operation (0 and 2 are each their own inverses).  Thus, $H$ \textbf{is} a subgroup of $\Z_4$.
\item Let $G$ be a group.  Then $\{e_G\}$ and $G$ are clearly both subgroups of $G$.\end{enumerate}
\end{example}

\begin{df}{Definitions} Let $G$ be a group. The subgroups $\{e_G\}$ and $G$ of
$G$ are called the \textit{trivial subgroup} and the \textit{improper
subgroup} of $G$, respectively. Not surprisingly, if $H\leq G$
and $H\neq \{e_G\}$, $H$ is called a \textit{nontrivial} subgroup
of $G$, and if $H\leq G$ and $H\neq G$, $H$ is called a \textit{proper subgroup} of $G$.\footnote{Sometimes the notation $H<G$
is used to indicated that $H$ is a proper subgroup of $G$, but
sometimes it is simply used to mean that $H$ is a
subgroup---proper or improper---of $G$.}\end{df}


Note that in the cases above, we saw subsets of groups fail to be subgroups because
they were not closed under the groups' operations; because they did not contain identity elements; or because they didn't contain an
inverse for each of their elements. None, however, failed  because the relevant group's operation was not associative on them.
This is not a coincidence: rather, since any element of a subset of a group $G$ also lives in $G$, any associative operation on $G$ is of
 necessity associative on any closed subset of $G$. Therefore, when we are checking to see if $H\cont G$ is a subgroup of group $G$, we need only check for
 closure, an identity element, and inverses.
\begin{lem}\label{subsame} Let $G$ be a group. \begin{enumerate}
\item
If $H$ is a subgroup of $G$ then the identity element $e_H$ of $H$
is $e_G$, the identity element of $G$.
\item If $H$ is a subgroup of $G$ and $a\in H$ has inverse $a^{-1}$ in $G$, then $a$'s inverse in $H$ is also $a^{-1}$.
\end{enumerate}\end{lem}

\begin{proof} For Part 1:  Since $e_H$ is in both $H$ and $G$, by the definition
    of $e_H$, we have
    $e_He_H=e_H$, and by the definition of $e_G$ we have $e_Ge_H=e_H$. So $e_He_H=e_Ge_H$, and thus by right
    cancellation, $e_H=e_G$.

    Next, for Part 2, let $b$ be the inverse of $a$ in $H$. Then using Part 1 of this lemma and the definition of an inverse, $ab=e_H=e_G=aa^{-1}$.  By left cancellation, then, we have that $b=a^{-1}$.
    \end{proof}

\begin{cor}Let $H\subseteq G$.  If the identity element of $G$ is not in $H$, then $H\not\leq G$.\end{cor}

\section{Proving that a subset of a group is or isn't a subgroup}

Using Lemma \ref{subsame} and the argument preceding it, we
have the following.
\begin{thm}\label{subgp} A subset $H$ of a group $G$ is a subgroup of $G$ if and only if
\begin{enumerate}
\item $H$ is closed under $G$'s operation;
\item The identity element of $G$ is in $H$; and
\item For each $a\in H$, $a$'s inverse in $G$ is contained in $H$. \end{enumerate}
\end{thm}

\begin{example}{subsetvsubgp2} For each of the following, prove that the given subset $H$ of group $G$ is or is not a subgroup of $G$.
\begin{enumerate}
\item $H=3\Z$, $G=\Z$.
\item $H=\{0,1,2,3\}$, $G=\Z_6$;
\item $H=\fatr^*$, $G=\fatr$;
\item $H=\{(0,x,y,z):x,y,z\in \fatr\}$, $G=\fatr^4$. \end{enumerate}
\end{example}

\begin{example}
Generalizing Part 1 of the above theorem, we have $n\Z\leq \Z$ for every $n\in \Z^+$. \red{The proof of this is left as an exercise for the reader.}\end{example}

\begin{example}{FHKL} Consider the group $\<F,+\>$, where $F$ is the set of all functions from $\fatr$ to $\fatr$ and $+$ is pointwise addition. Which of the following are subgroups of $F$?
\begin{enumerate}
\item[] $H=\{f\in F: f(5)=0\}$;
\item[] $K=\{f\in F: f \mbox{ is continuous}\}$;
\item[] $L=\{f\in F: f \mbox{ is differentiable}\}$.
\end{enumerate}
Are any of $H$, $K$, and $L$ subgroups of one another? \end{example}

 In fact, we can narrow down the number of facts we need to
check to prove a subset $H\cont G$ is a subgroup of $G$ to only
two.

\begin{thm}\label{twostep} \textbf{Two-Step Subgroup Test.} Let $G$ be
a group and $H\cont G$.  Then $H$ is a subgroup of $G$ if
\begin{center}
\begin{tabular}{ll}
1.& $H\neq \emptyset$; and\\
2.& For each $a,b\in H$, $ab^{-1}\in H$.
\end{tabular}
\end{center}
\end{thm}

\begin{proof} Assume that the above two properties hold.   Since $H\neq
\emptyset$, there exists an $x\in G$ such that $x\in H$.  Then
$e_G=xx^{-1}$ is in $H$, by the second property. Next, for every
$a\in H$ we have $a^{-1}=e_Ga^{-1}\in H$ (again by the second
property). Finally, if $a,b\in H$ then we've already shown
$b^{-1}\in H$; so $ab=a(b^{-1})^{-1}\in H$, yet again by the second
property.  Thus, $H\leq G$.\end{proof}

\begin{example}{}\
 \begin{enumerate}\item Use the Two-Step Subgroup Test to prove that $3\Z$ is a subgroup of $\Z$.
\item Use the Two-Step Subgroup Test to prove that $SL(n,\fatr)$ is a subgroup of $GL(n,\fatr)$. \end{enumerate}
\end{example}

\begin{df} Note that if $H$ is a subgroup of a group $G$ and $K$ is a subset of $H$, then $K$ is a subgroup of
  $H$ if and only if it's a subgroup of $G$.\end{df}


It can be useful to look at how subgroups of a group relate to one
another.  One way of doing this is to consider \textit{subgroup
lattices} (also known as \textit{subgroup diagrams}.  To draw a
subgroup lattice for a group $G$, we list all the subgroups of $G$,
writing a subgroup $K$ below a subgroup $H$, and connecting them
with a line, if $K$ is a subgroup of $H$.

\begin{example}{} Consider the group $\Z_8$. We will see later that the subgroups of $\Z_8$ are $\{0\}$, $\{0,2,4,6\}$, $\{0,4\}$ and $\Z_8$ itself.  So $\Z_8$ has the following subgroup lattice.

$$\xymatrix{&\Z_8 \ar@{-}[d]&\\ &\{0,2,4,6\}\ar@{-}[d]&\\ &\{0,4\} \ar@{-}[d]&\\ &\{0\} &} $$
\end{example}


\begin{example}{} Referring to Example \ref{FHKL}, draw the portion of the subgroup lattice for $F$ that shows the relationships between itself and its proper subgroups $H$, $K$, and $L$. \end{example}


\begin{example}{} Indicate the subgroup relationships between the following groups:
$\Z$, $12\Z$, $\fatq^+$, $\fatr$, $6\Z$, $\fatr^+$, $3\Z$, $G=\<\{\pi^n:n\in \Z\},\cdot\,\>$
and
$J=\<\{6^n:n\in \Z\},\cdot\,\>. $
\end{example}

 We end with a theorem about homomorphisms and subgroups
that leads us to another group invariant.
\begin{thm}\label{imsubgp} Let $G$ and $G'$ be groups, let $\phi$ a homomorphism from $G$ to $G'$, and let $H$ a subgroup of $G$.  Then $\phi(H)$ is a subgroup of $G'$.
\end{thm}

\begin{proof} \red{This proof is left as an exercise for the reader.}\end{proof}

\begin{cor}\label{} If $G\simeq G'$ and $G$ contains exactly $n$ subgroups ($n\in \Z^+$), then so does $G'$.
\end{cor}


This is another way of, for instance, distinguishing between the groups $\Z_4$ and the Klein 4-group $Z_2^2$.


\begin{example}{} By inspection, $\Z_4$ has subgroup lattice
$$\xymatrix{\Z_4 \ar@{-}[d]\\ \{0,2\}\ar@{-}[d]\\ \{0\}} $$ and $\Z_2^2$ has subgroup lattice
$$\xymatrix{&\Z_2^2 \ar@{-}[dl]\ar@{-}[d]\ar@{-}[dr]&\\ \{(0,0),(1,0)\}\ar@{-}[dr]&\{(0,0),(1,1)\}\ar@{-}[d]&\{(0,0),(0,1)\}\ar@{-}[dl]\\ &\{(0,0)\}. &}$$
Since $\Z_4$ contains exactly 3 subgroups and $\Z_2^2$ exactly 5, we have
that $\Z_4\not\simeq V$. \end{example}


\pagebreak
\section{Exercises}


\begin{exercise}[ID=4A]
\tf Throughout, let $G$ and $G'$ be  groups.

\begin{enumerate}

\item Every group contains at least two distinct subgroups.

\item If $H$ is a proper subgroup of group $G$ and $G$ is finite, then we must have $|H|<|G|$.

\item $7\Z$ is a subgroup of $14\Z$.

\item A group $G$ may have two distinct proper subgroups which are isomorphic (to one another).

\end{enumerate}
\end{exercise}

\begin{solution}[print=false]

\begin{inparaenum}[(a)]
\item F \hfill \item T \hfill \item F \hfill  \item T
\end{inparaenum}

\end{solution}


\begin{exercise}[ID=4B]
Give specific, precise examples of the following groups $G$ with subgroups $H$:
\begin{enumerate}
\item A group $G$ with a proper subgroup $H$ of $G$ such that $|H|=|G|$.
\item A group $G$ of order 12 containing a subgroup $H$ with $|H|=3$.
\item A nonabelian group $G$ containing a nontrivial abelian subgroup $H$.
\item A finite subgroup $H$ of an infinite group $G$.
\end{enumerate}

\end{exercise}

\begin{solution}[print=false]
(Other answers are possible.)


\begin{enumerate}
\item $G=\Z$, $H=2\Z$
\item $G=\Z_{12}$, $H=\{0,4,8\}$
\item $G=GL(2,\fatr)$, $H=\{\pm I_2\}$
\item $G=\fatr^*, H=\{\pm 1\}$.
\end{enumerate}


\end{solution}

\begin{exercise}
Let $n\in \Z^+$.
\begin{enumerate}
\item Prove that $n\Z \leq \Z$.
\item Prove that the set $H=\{A\in \M_n(\fatr)\,:\,\det A=\pm 1\}$ is a subgroup of $GL(n,\fatr)$.
\end{enumerate}
(\textbf{Note:} Your proofs do not need to be long to be correct!)
\end{exercise}

\begin{solution}[print=false]
\begin{enumerate}
\item We know that $n\Z\subseteq \Z$ and that $n\Z$ is a group under addition, the group operation in $\Z$.  Therefore, $n\Z\leq \Z$.
\item Clearly, $H$ is a nonempty subset of $GL(n,\fatr)$ (for instance, $I_n\in H$).  Next, let $A,B\in H$.
Then $$\det(AB^{-1})=(\det A)(\det (B^{-1}))=(\pm 1)\left(\frac{1}{\pm 1}\right)=\pm 1,$$ so $AB^{-1}\in H$.  Thus, $H\leq GL(n,\fatr)$ by the Two-Step Subgroup Test.
\end{enumerate}
\end{solution}


\begin{exercise}
For each group $G$ and subset $H$, decide whether or not $H$ is a subgroup of $G$. In the cases in which $H$ is \underline{not} a subgroup of $G$, provide a proof. (\textbf{Note.} Your proofs do not need to be long to be correct!)
\begin{enumerate}
\item $G=\Z$, $H=\fatr$
\item $G=\Z_{15}$, $H=\{0,5,10\}$
\item $G=\Z_{15}$, $H=\{0,4,8,12\}$
\item $G=\fatc$, $H=\fatr^*$
\item $G=\fatc^*$, $H=\{1,i,-1,-i\}$
\item $G=\M_n(\fatr)$, $H=GL(n,\fatr)$
\item $G=GL(n,\fatr)$, $H=\{A\in \M_n(\fatr)\,:\,\det A = -1\}$
\end{enumerate}
\end{exercise}

\begin{solution}[print=false]
\begin{enumerate}
\item $H\not\subseteq G$, so $H\not\leq G$.
\item $H\leq G$.
\item $H$ is a subset of $G$ but isn't closed under $+_{15}$: for instance $12+_{15}4=1\not\in H$.  So $H\not\leq G$.
\item $H$ is a subset of $G$, but the identity element, 0, of $G$ isn't in $H$, so $H\not\leq G$.
\item $H\leq G$.
\item $H$ is a subset of $G$, but the identity element, $\0$, of $G$ isn't in $H$, so $H\not\leq G$.
\item $H$ is a subset of $G$, but the identity element, $I_n$, of $G$ isn't in $H$, since $\det I_n=1$.  Thus, $H\not\leq G$.
\end{enumerate}
\end{solution}


\begin{comment}
\begin{exercise}[ID=4C]

\begin{enumerate}
\item
Draw the subgroup lattices for the groups $\Z_2 \times \Z_3$ and $\Z_2 \times \Z_4$. (Identify each subgroup in the following lattices explicitly, either by listing its elements within set brackets, or by otherwise identifying it using precise mathematical notation.)

\item Would it be reasonable for me to ask you to draw the complete subgroup lattice for $\Z$?  If so, draw it; otherwise, explain why that's an unreasonable request. \textbf{Make sure any sentences you write are grammatically correct.}

\end{enumerate}
\end{exercise}

\begin{solution}[print=false]
\end{solution}

\end{comment}

\begin{exercise}
 Let $G$ and $G'$ be groups, let $\phi$ a homomorphism from $G$ to $G'$, and let $H$ a subgroup of $G$.  Then $\phi(H)$ is a subgroup of $G'$.
 \end{exercise}

\begin{solution}[print=false]
Let $x,y\in \phi(H)$.  Then $x=\phi(a)$ and $y=\phi(b)$ for some $a,b\in H$.  So $xy=\phi(a)\phi(b)=\phi(ab)$ (since $\phi$ is a homomorphism).  Since $H\leq G$, $ab\in H$.  Thus, $xy\in \phi(H)$.  Next, $e_G\in H$, so $e_{G'}=\phi(e_G)\in \phi(H)$. Finally, since $H\leq G$ and $a\in H$, $a^{-1}\in H$; so $x^{-1}=\phi(a)^{-1}=\phi(a^{-1}) \in \phi(H)$. Thus, $\phi(H)\leq G'$.

\end{solution}

\begin{exercise}[ID=4D, subtitle=(Extra Credit)]
Recall that an element $u$ of a binary structure is said to be an \textit{idempotent} if $u^2=u$.  Let $G$ be an abelian group, and let $U$ be the set of all idempotents of $G$.  Prove that $U$ is a subgroup of $G$.
\end{exercise}

\begin{solution}[print=false]
Clearly, $e\in H$, so $H\neq \emptyset$.  Next, let $u,v\in H$. Then
\begin{align*}(uv^{-1})^2&=(uv^{-1})(uv^{-1})&&\\
&=u^2(v^{-1})^2 &&\text{(since the group operation is assoc. and comm.)}\\
&=u^2(v^2)^{-1} &&\text{(using the rules for exponents)}\\
&=uv^{-1}&&\text{(since $u,v\in H$).}
\end{align*}
Thus, $uv^{-1}\in H$, and so $H \leq G$ by the Two-Step Subgroup Test. \end{solution}

