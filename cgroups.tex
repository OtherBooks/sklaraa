\chapter{Cyclic Groups}\label{cyc}%\footnote{See Sections 5--6 in \cite{F}.}

\section{Introduction to cyclic groups}
Certain groups and subgroups of groups have particularly nice structures.

\begin{df}{Definition} A group is \textit{cyclic} if it is isomorphic to $\Z_n$ for some $n\geq 1$, or if it is isomorphic to $\Z$.\end{df}


\begin{example}{} Examples/nonexamples of cyclic groups.
\begin{enumerate}
\item $n\Z$ and $\Z_n$ are cyclic for every $n\in \Z^+$.
\item $\fatr$, $\fatr^*$, $\M_2(\fatr)$, and $GL(2,\fatr)$ are uncountable and hence can't be cyclic.
\item $\Z_2^2$ is \underline{not} cyclic since it would have to be isomorphic to $\Z_4$ if it were (since it has order 4).

 \item $\fatq$ isn't cyclic. If it were cyclic it would have to be isomorphic to $\Z$, since $\fatq$
 is an infinite group (so can't be isomorphic to $\Z_n$ for any $n$). But we showed in Example \ref{zq} that $\fatq \not\simeq \Z$.

\end{enumerate}
\end{example}


\begin{df}{Remark} Clearly, cyclicity is a group invariant.\end{df}

\begin{thm}\label{abcyc} If $G$ is cyclic, then $G$ is abelian; however, $G$
can be abelian but not cyclic.
\end{thm}

\begin{proof} Since $\Z$ and each $\Z_n$ are abelian, every cyclic group is abelian.  But, for instance, $\fatr$ is an abelian noncyclic group (it can't be cyclic because it is uncountable).\end{proof}

\begin{df}{Definition} Let $a$ be an element of a group $G$.  Then
$$\<a\> = \{a^n:n\in \Z\}=\{\ldots, a^{-2},a^{-1},a^0,a^1,a^2,\ldots\}$$is called the \textit{(cyclic) subgroup of $G$ generated by $a$}.
 (We will later see why the words ``cyclic" and
``generated" are used here.)\end{df}

\begin{df}{Remark} Note that in the above definition, we were using
multiplicative notation.  Using additive notation, we have
$\<a\>=\{na:n\in \Z\}=\{\ldots, -2a, -a, 0a, 1a, 2a, \ldots\}.$ \end{df}


\begin{thm}\label{} Let $a$ be an element of a group $G$.  Then $\<a\>\leq G$.
\end{thm}

\begin{proof} Let $x,y\in \<a\>$.  Then $x=a^i$, $y=a^j$ for some $i,j\in \Z$.  So $xy=a^ia^j=a^{i+j}\in \<a\>$.  Next, $e_G=a^0\in H$. Finally, $(a^i)^{-1}=a^{-i}\in H$. So $\<a\>$ is a subgroup of $G$.\end{proof}




\begin{thm}\label{} Let $G$ be a group with identity element $e$, and let $a\in G$.  Then the group $\<a\>$ is cyclic.\end{thm}

\begin{proof}%FIX

\textbf{Case 1.} There is no positive integer $k$ with $a^k=e$. In this case, we claim $\<a\>\simeq \Z$. Indeed: Let $\phi:\Z\to \<a\>$ be defined by $\phi(i)=a^i$. Clearly, $\phi$ is a homomorphism that is onto. So to show that $\phi$ is an isomorphism, it suffices to show that $\phi$ is one-to-one.  Let $i,j\in \Z$ with $\phi(i)=\phi(j)$.  Then $a^i=a^j$.  Without loss of generality, we may assume $i\geq j$.  Then, multiplying both sides of the equation by $a^{-j}$ (on the right or on the left) we obtain $a^{i-j}=e$.  Since $i-j\in \fatn$ and there is no positive integer $k$ with $a^k=e$, we must have $i-j=0$, so $i=j$. Thus, $\phi$ is an isomorphism from $\Z$ to $\<a\>$.

\textbf{Case 2.} There is a positive integer $k$ such that $a^k=e$.  In this case, let $n$ be the \underline{least} such positive integer. In this case, we claim $\<a\>\simeq \Z_n$. Indeed: Let $\phi:\Z_n\to \<a\>$ be defined by $\phi(i)=a^i$. We will find it useful to use the following, which holds in this context:


\begin{lem}\label{}\textbf{(Clock Lemma)} If $s,t\in \Z$, then $a^s=a^t$ if and only if $s$ is congruent to $t$ modulo $n$.\end{lem}

\begin{proof}By the Division Algorithm, there exist $q\in \Z$ and $r\in \Z_n$ with $s-t=qn+r$. Then $a^{s-t}=a^{qn+r}=(a^n)^qa^r=e^qa^r=a^r$. Since $0\leq r<n$, be definition of $n$ we have that $a^{s-t}=e$ if and only if $r=0$; that is, $a^s=a^t$ if and only if $s$ and $t$ are congruent modulo $n$.\end{proof}


Let $i,j\in \Z$.  We want to show that $\phi(i+_nj)=\phi(i)\phi(j)$, that is, that $a^{i+_nj}=a^ia^j$. Since $i+j$ is congruent to $i+_nj$ modulo $n$, $a^ia^j=a^{i+j}=a^{i+_nj}$, by the Clock Lemma. So $\phi$ is a homomorphism. Next we show that $\phi$ is one-to-one.  Let $i,j\in \Z_n$ with $\phi(i)=\phi(j)$, so $a^i=a^j$.  Then, by the Clock Lemma, $i$ and $j$ are congruent modulo $n$. Since they're both in $\Z_n$, they must be equal,
so $\phi$ is one-to-one. Finally, we show that $\phi$ is onto. let $x\in \<a\>$.  Then $x=a^i$ for some $i\in \Z$. Letting $r$ be the remainder when we divide $i$ by $n$, we have $r\in \Z_n$ with $i$ congruent to $r$ modulo $n$; so, again using the Clock Lemma, $x=a^i=a^r$. Since $r\in \Z_n$, $x=\phi(r)$. So $\phi$ is onto. Thus, $\phi$ is an isomorphism.\end{proof}



\begin{df}{Definition and notation} Let $G$ be a group and let $a\in G$. We define the \textit{order} of $a$, denoted $o(a)$, to be $|\<a\>|$. (Note: If there exists a positive integer $k$ such that $a^k=e$, then the least such integer is the order of $a$; otherwise, $o(a)=\infty$.)\end{df}

\begin{df}{Remark} Do not confuse the order of a group with the order
of an \underline{element of a group}. These are related concepts, but they are distinct, and have distinct
notations: as we've seen, the order of a group, $G$, is denoted by
$|G|$, while the order of an element $a$ of a group is denoted by
$o(a)$.\end{df}

We have the following handy theorem.

\begin{thm}\label{invord}
Let $a\in G$.  Then $o(a)=o(a^{-1})$.
\end{thm}

\begin{proof}
First, assume $o(a)=n<\infty$.  Then $$(a^{-1})^n=(a^n)^{-1}=e^{-1}=e,$$ so $o(a^{-1})\leq n=o(a)$.  Using the same argument, we have $n=o(a)\leq o(a^{-1})$. Since $o(a^{-1})\leq n$ and $n\leq o(a^{-1})$, $o(a^{-1})=n=o(a)$.

On the other hand, assume $o(a)= \infty$.  Then $a^{-1}$ must also have infinite order, since if it had finite order $m$, $a$ would, by the above argument, have order less than or equal to $m$.
\end{proof}

Unfortunately, there's no formula one can simply use to compute the order
of an element in an arbitrary group. However, in the special case
that the group is cyclic of order $n$, we do have such a formula. We
present the following result without proof.

\begin{thm}\label{znorders} For each $a\in \Z_n$, $o(a)=n/\gcd(n,a)$. \end{thm}

Here are some examples of cyclic subgroups of groups, and orders of group elements.

\begin{example}{csexs}\

\begin{enumerate}
\item In $\Z$, $\<2\>=\{\ldots,-4,-2,0,2,4,\ldots\}=2\Z$. More generally, given any $n\in \Z$, in $\Z$ we have $\<n\>=n\Z$. For $a\in \Z$, $o(a)=\infty$ if $a\neq 0$; $o(0)=1$.

\item In $\Z_8$, we have $\<0\>=\{0\}$, $\<1\>=\<3\>=\<5\>=\<7\>=\Z_8,$ $\<2\>=\<6\>=\{0,2,4,6\},$ and $\<4\>=\{0,4\}$.
\item In $\fatr$, $\<\pi\>=\pi\Z$, so $o(\pi)=\infty$.
\item In $\fatr^*$, $\<\pi\>=\{\pi^n:n\in \Z\}$. Again, $o(\pi)=\infty$.
\item In $\M_2(\fatr)$,
$$\left\<\left[
    \begin{array}{cc}
      1 & 1 \\
      0 & 1 \\
    \end{array}
  \right]\right\>=\left\{\left[
    \begin{array}{cc}
      c & c \\
      0 & c \\
    \end{array}
  \right] : c\in \Z \right\}= \left\{c\left[
    \begin{array}{cc}
     1 & 1 \\
      0 & 1 \\
    \end{array}
  \right] : c\in \Z \right\}.$$ The order of the the matrix is therefore infinity.
\item In $\M_2(\Z_2)$, if $A=\left[
    \begin{array}{cc}
      1 & 1 \\
      0 & 1 \\
    \end{array}
  \right]$, then $\<A\>=\{A, \0\}$, so $A$ has order 2.
\item In $GL(2,\Z_2)$, $\left[
    \begin{array}{cc}
      1 & 1 \\
      0 & 1 \\
    \end{array}
  \right]$ has order 2. (Why?)
\item In $GL(2,\fatr)$, $\left[
    \begin{array}{cr}
      0 & -1 \\
      1 & 0 \\
    \end{array}
  \right]$ has order 4. (Why?)
\item In $GL(2,\fatr)$, $\left[
    \begin{array}{cc}
      1 & 1 \\
      0 & 1 \\
    \end{array}
  \right]$ has infinite order. (Why?)

\item In $GL(4,\Z_2)$,
 $A=\left[
      \begin{array}{cccc}
        0 & 0 & 1 & 0 \\
        0 & 0 & 0 & 1 \\
        1 & 0 & 0 & 0 \\
        0 & 1 & 0 & 0 \\
      \end{array}
    \right]$ has order 2. (Why?) \end{enumerate}
\end{example}

\begin{thm}\label{} Every cyclic group $G$ is of the form $\<a\>$ for some $a\in G$.\end{thm}

\begin{proof} Let $G$ be cyclic. Suppose $|G|=\infty$. Then there is an isomorphism $\phi: \Z\to G$.  Note that $\Z=\<1\>$. So
$$G=\phi(\Z)=\{\phi(a)\,:\,a\in \Z\}=\{\phi(a(1))\,:\,a\in \Z\}=\{a\phi(1)\,:\,a\in \Z\}=\<\phi(1)\>.$$ A similar argument shows that there exists $a\in G$ such that $G=\<a\>$ when $|G|<\infty$.\end{proof}

\begin{df}{Definitions} Let $G$ be a group. An element $a\in G$ is a \textit{generator of $G$} (equivalently, $a$ \textit{generates $G$}) if
$\<a\>=G$.\end{df}

\begin{df}{Remarks}\

 \begin{enumerate}
\item Note that if $G$ has a generator, then it is necessarily a cyclic group.
\item Note that an element $a$ of a group $G$ generates $G$ if and only if every element of $G$ is of the form $a^n$ for some $n\in \Z$.
\item Generators of groups \underline{need not} be unique.  For instance, we saw in Example \ref{csexs} that each of the elements $1,3,5$ and $7$ of $\Z_8$ is a generator for $\Z_8$.
\end{enumerate}
\end{df}

\begin{example}{}
\hspace{10pt}

\begin{enumerate}
\item $\Z$ has generator $1$ and generator $-1$.
\item Given $n\in \Z$, $nZ$ has generator $n$ and generator $-n$.
\item
Given $n\geq 2$ in $\Z$, the generators of $\Z_n$ are exactly  the elements $a\in \Z_n$ such that $\gcd(n,a)=1$. (This follows from Theorem \ref{znorders}.) \end{enumerate}
\end{example}


\begin{df}{Remark}
Order of elements provides another group invariant.\end{df}

\begin{thm}\label{ophia} Let $\phi:G\to G'$ be a group isomorphism and let $a\in G$.  Then $o(\phi(a))=o(a)$.
\end{thm}

\begin{proof} This follows from the fact that $\phi(\<a\>)=\<\phi(a)\>$.\end{proof}


\begin{cor}\label{} If groups $G$ and $G'$ are isomorphic then for any $n\in \Z^+$, the number of elements of $G$ of order $n$ is the same as the number of elements of $G'$ of order $n$. \end{cor}


\begin{example}{} Since $1$ in $\Z_4$ has order 4 but every element in $\Z_2 \times \Z_2$ has order less than or equal to 2, these groups cannot be isomorphic. \end{example}


\warn{Note, however, that just because the orders of elements of two groups ``match up" the groups need not be isomorphic.  For example, every element of $\Z$ has infinite order, except for its identity element, which has order 1; the same is true for the group $\fatq$.  However, we have previously proven that these groups are not isomorphic.}

\bigskip
\section{Exploring the subgroup lattices of cyclic groups}

We now explore the subgroups of cyclic groups. A complete proof of the following theorem is provided on p. 61 of \cite{F}.

\begin{thm}\label{subc} Every subgroup of a cyclic group is cyclic.
\end{thm}

\textbf{Sketch of proof:} Let $G=\<a\>$ and $H\leq G$. If $H=\{e\}$, then clearly $H$ is cyclic.  Else, there exists an element $a^i$ in $H$ with $i>0$; let $d$ be the least positive integer such that $a^d\in H$. It turns out that $H=\<a^d\>$.
\bigskip
\begin{cor}\label{} Every subgroup of $\Z$ is of the form $n\Z$ for $n\in \Z$. (Note that $n\Z\simeq \Z$ unless $n=0$.) \end{cor}

Really, it suffices to study the subgroups of $\Z$ and $\Z_n$ to understand the subgroup lattice of every cyclic group.

We provide the following theorem without proof.

\begin{thm}\label{znsubgps} The nontrivial subgroups of $\Z_n$ are exactly those of the form $\<d\>$, where $d$ is a positive divisor of $n$. Note that $|\<d\>|=n/d$ for each such $d$.\footnote{
\textbf{Remark.} It turns out that for each $0\neq a\in \Z_n$, $\d{\<a\>=\left\<\frac{n}{\gcd(n,a)}\right\>}$. Note that it follows that $|\<a\>|=n/\gcd(n,a)$ for every $0\neq a \in \Z_n$.} \end{thm}


In fact:
\begin{thm}\label{} $\Z_n$ has a unique subgroup of order $k$, for each positive divisor $k$ of $n$.\end{thm}

\begin{example}{} How many subgroups does $\Z_{18}$ have?  What are the generators of $\Z_{15}$?

Well, the positive divisors of $18$ are $1,2,3,6,9,$ and $18$, so $\Z_{18}$ has exactly six subgroups (namely, $\<1\>$, $\<2\>$, etc.).
The generators of $\Z_{15}$ are the elements of $\Z_{15}$ that are relatively prime to $15$, namely $1,2,4,7,8,11,13,$ and $14$. \end{example}


\begin{example}{} Draw a subgroup lattice for $\Z_{12}$.

The positive divisors of $12$ are $1,2,3,4,6,$ and $12$; so
$\Z_{12}$'s subgroups are of the form $\<1\>$, $\<2\>$, etc. So
$\Z_{12}$ has the following subgroup lattice.
$$\xymatrix{
&&\Z_{12} \ar@{-}[ld] \ar@{-}[rd]&\\
&\{2,4,6,8,10,0\} \ar@{-}[ld]\ar@{-}[rd]&&\{3,6,9,0\} \ar@{-}[ld]\\ \{4,8,0\}\ar@{-}[dr]&& \{6,0\} \ar@{-}[dl]& \\
& \{0\}&&}$$
\end{example}

\pagebreak
\section{Exercises}

\begin{exercise}
\tf Throughout, let $G$ be a group with identity element $e$.

\begin{enumerate}
\item If $G$ is infinite and cyclic, then $G$ must have infinitely many generators.
\item There may be two distinct elements $a$ and $b$ of a group $G$ with $\<a\>=\<b\>$.
\item If $a,b\in G$ and $a\in \<b\>$ then we must have $b\in \<a\>$.
\item If $a\in G$ with $a^4=e$, then $o(a)$ must equal $4$.
\item If $G$ is countable then $G$ must be cyclic.
\end{enumerate}
\end{exercise}

\begin{solution}[print=true]

\begin{inparaenum}[(a)]
\item F \hfill \item T \hfill \item  F \hfill \item F \hfill \item F
\end{inparaenum}

\end{solution}


\begin{exercise}
Give examples of the following.

\begin{enumerate}
\item An infinite noncyclic group $G$ containing an infinite cyclic subgroup $H$.
\item An infinite noncyclic group $G$ containing a finite nontrivial cyclic subgroup $H$.
\item A cyclic group $G$ containing exactly 20 elements.
\item A nontrivial cyclic group $G$ whose elements are all matrices.
\item A noncyclic group $G$ such that every proper subgroup of $G$ is cyclic.
\end{enumerate}
\end{exercise}

\begin{solution}[print=true]
(Other answers are possible.)

\begin{enumerate}
\item $G=\fatr$, $H=\Z$ \quad
\item $G=GL(n,\fatr)$, $H=\{\pm I_2\}$
\item $G=\Z_{20}$
\item $G=\{\pm I_2\}$, under matrix multiplication
\item $G=\Z_2^2$
\end{enumerate}
\end{solution}

\begin{exercise}
Find the orders of the following elements in the given groups.

\begin{enumerate}
\item $2\in \Z$
\item $-i\in \fatc^*$
\item $-I_2\in GL(2,\fatr)$
\item $-I_2\in \M_2(\fatr)$
\item $(6,8)\in \Z_{10}\times \Z_{10}$
\end{enumerate}
\end{exercise}

\begin{solution}[print=true]
\begin{enumerate}
\item $o(2)=\infty$, since $\<2\>=2\Z$.
\item $o(-i)=4$, since $\<-i\>=\{-i,-1,i,1\}$.
\item $o(I_2)=2$, since $\<-I_2\>=\{-I_2,I_2\}$.
\item $o(I_2)=\infty$, since $\<-I_2\>=\{kI_2\,:\,k\in \Z\}$.
\item $o((6,8))=5$, since $\<(6,8)\>=\{(6,8),(2,6),(8,4),(4,2),(0,0)\}$.
\end{enumerate}
\end{solution}

\begin{exercise}
For each of the following, if the group is cyclic, list \textit{all} of its generators. If the group is \textit{not} cyclic, write NC.

\medskip

\noindent
\begin{inparaenum}[(a)]
\item $5\Z$ \hfill
\item $\Z_{18}$ \hfill
\item $\fatr$ \hfill
\item $\<\pi\>$ in $\fatr$ \hfill
\item $\Z_2^2$ \hfill
\item $\<8\>$ in $\fatq^*$
\end{inparaenum}

\end{exercise}

\begin{solution}[print=true]

\begin{inparaenum}[(a)]
\item $\pm 5$ \hfill
\item $1,5,7,11,13,17$ \hfill
\item NC \hfill
\item $\pm \pi$ \hfill
\item NC \hfill
\item $8,1/8$
\end{inparaenum}

\end{solution}


\begin{exercise}
Explicitly identify the elements of the following subgroups of the given groups. You may use set-builder notation if the subgroup is infinite, or a conventional name for the subgroup if we have one.

\begin{enumerate}
\item $\<3\>$ in $\Z$
\item $\<i\>$ in $C^*$
\item $\<A\>$, for $A=\left[ \begin{array}{cc}
                          1 & 0 \\
                          0 & 0 \\
                        \end{array}
                      \right]\in \M_2(\fatr)$
\item $\<(2,3)\>$ in $\Z_4\times \Z_5$
\item $\<B\>$, for $B=\left[ \begin{array}{cc}
                          1 & 1\\
                          0 & 1 \\
                        \end{array}
                      \right]\in GL(2,\fatr)$
\end{enumerate}

\end{exercise}


\begin{solution}[print=true]

\begin{enumerate}
\item $3\Z$
\item $\{i,-1,-i,1\}$
\item $\left\{\left[
                \begin{array}{cc}
                  k & 0 \\
                  0 & 0 \\
                \end{array}
              \right]\,:\,k\in \Z
\right\}$
\item $\{(2,3),(0,1),(2,4),(0,2),(2,0),(0,3),(2,1),(0,4),(2,2),(0,0)\}$
\item $\left\{\left[
                \begin{array}{cc}
                  1 & k \\
                  0 & 1 \\
                \end{array}
              \right]\,:\,k\in \Z\right\}$
\end{enumerate}

\end{solution}

\begin{exercise}
Draw subgroup lattices for \begin{inparaenum}[(a)]
\item $\Z_6$,
\item $\Z_{13}$, and
\item $\Z_{18}$.
\end{inparaenum}
\end{exercise}

\begin{solution}[print=true]
Solutions for (a)--(c) are shown below.

$$\begin{array}{lll}\xymatrix{&\Z_6&\\\<2\>\ar@{-}[ur]&&\<3\>\ar@{-}[ul]\\&\<0\>\ar@{-}[ul]\ar@{-}[ur]&}& \hspace{50pt}
\xymatrix{\Z_{13}\ar@{-}[d]\\ \<0\>}& \hspace{50pt}  \xymatrix{&\Z_{18}&&\\\<2\>\ar@{-}[ur]\ar@{-}[dr]&&\<3\>\ar@{-}[ul]\ar@{-}[dl]&\\
&\<6\>\ar@{-}[dr]\ar@{-}[ur]&&\<9\>\ar@{-}[dl]\ar@{-}[ul]\\&&\<0\>&}\end{array}$$
\end{solution}

\begin{exercise}
Let $G$ be a group with no nontrivial proper subgroups.  Prove that $G$ is cyclic.
\end{exercise}

\begin{solution}[print=true]
 Let $e$ be the identity
    element of $G$. If $G=\{e\}$, then $G$ is clearly
    cyclic. Else, there exists $a\neq e$ in $G$. Then
    $\<a\>$ is a subgroup of $G$.  Since $e\neq a\in
    \<a\>$, $\<a\>$ is a nontrivial subgroup of $G$. Therefore, we must have
    $\<a\>=G$. Hence, $G$ is cyclic, as desired.
\end{solution}