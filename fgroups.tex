\chapter{Factor Groups}\label{factorgps}%\footnote{See Section 14 in \cite{F}.}

\section{Motivation}

We mentioned previously that given a subgroup $H$ of $G$, we'd like
to use $H$ to get at some understanding of $G$'s entire structure.
Recall that we've defined $G/H$ to be the set of all left cosets of
$H$ in $G$.  What we'd like to do now is equip $G/H$ with some
operation under which $G/H$ is a group!  The natural way to do this
would be to define multiplication on $G/H$ by
$$(aH)(bH)=abH \mbox{ for all }a,b\in G.$$
Ok, so let's do that!  But wait: we're defining this operation
by referring to coset representatives, so we'd better make sure
our operation is well defined. Only it sadly turns out that in
general this operation is \underline{not} well defined. For
example:


\begin{example}{} Let $H=\<(12)\>$ in $S_3$, and let $a=(13)$ and
$b=(132)$.  Let $x=aH$ and $y=bH$ in $S_3/H$. Then using the
above-defined operation on $G/H$ we'd have
$$xy=(aH)(bH)=abH=(13)(132)H=(23)H.$$ But also $x=(123)H$ and $y=(23)H$, so
we'd also have
$$xy=((123)H)((23)H))=(123)(23)H=(12)H=H\neq (23)H.$$
So this operation isn't well defined. \end{example}

 The question for us now becomes: what conditions must hold
for $H$ in $G$ in order for operation $$(aH)(bH)=abH$$ on $G/H$
to be well defined?  It turns out that this operation is well
defined exactly when $H$ is normal in $G$! We state this in the
following theorem:

\begin{thm}\label{} Let $H\leq G$.  Then the operation $$(aH)(bH)=abH$$ on
$G/H$ is well defined if and only if $H \unlhd G$. \end{thm}

\begin{proof}

($\Leftarrow$)  Let $a_1,a_2,b_1,b_2\in G$ with $a_1H=a_2H$ and
$b_1H=b_2H$.  We want to show that then $a_1b_1H=a_2b_2H$, that is,
that $(a_2b_2)^{-1}a_1b_1\in H$. Well, since $a_1H=a_2H$ we have
$a_2^{-1}a_1\in H$. So
$$(a_2b_2)^{-1}a_1b_1=b_2^{-1}(a_2^{-1}a_1)b_1 \in b_2^{-1}Hb_1.$$ Next, since $H\unlhd G$,
we have $Hb_1=b_1H$, so $b_2^{-1}Hb_1=b_2^{-1}b_1H$; but since
$b_1H=b_2H$, we have $b_2^{-1}b_1\in H$, so $b_2^{-1}b_1H=H$.  Thus,
$(a_2b_2)^{-1}a_1b_1\in H$, as desired.

($\Rightarrow$) Let $a\in G$.  We want to show that $aH=Ha$.  Well,
first let $x\in aH$.  Then $$(xH)(a^{-1}H)=xa^{-1}H$$ and, since
$xH=aH$,
$$(xH)(a^{-1}H)=(aH)(a^{-1}H)=aa^{-1}H=H.$$  Since our operation on $G/H$ is well defined, this means that
$xa^{-1}H=H$, so $xa^{-1}\in H$; thus, $x\in Ha$.  We conclude that
$aH\subseteq Ha$.  The proof that $Ha\subseteq aH$ is similar.\end{proof}

\begin{df}{Definition} When $H\unlhd G$, the well defined operation
$(aH)(bH)=abH$ on $G/H$ is called \textit{left coset
multiplication}.\end{df}

  It is clear that normal subgroups will be
very important for us in studying group structures.  We therefore
spend some time looking at normal subgroups before returning to
equipping $G/H$ with a group structure.

\section{Focusing on normal subgroups}

We first provide a theorem that will help us in identifying when a
subgroup of a group is normal. First, we provide a  definition.

\begin{df}{Definition} Let $H$ be a subgroup of $G$ and $a,b$ in $G$. We define
$$aHb=\{ahb\,:h\in H\}.$$\end{df}

\begin{thm}\label{norm.thm} Let $H$ be a subgroup of a group $G$.  Then the
following are equivalent:
\begin{enumerate}
\item $H$ is normal in $G$;
\item $aHa^{-1}=H$ for all $a\in G$;
\item $aHa^{-1}\subseteq H$ for all $a\in G$.
\end{enumerate}
\end{thm}

\begin{proof} The equivalence of Statements 1 and 2 is clear, as is the
fact that Statement 2 implies Statement 3.  So it suffices to show
that Statement 3 implies Statement 2. Well, assume that
$aHa^{-1}\subseteq H$ for all $a\in G$, and let $b\in G$.  We want
to show that $bHb^{-1}=H$. Since Statement 3 holds, we clearly have
$bHb^{-1}\subseteq H$.  But, again using Statement 3, we also have
$b^{-1}Hb\subseteq H$; multiplying both sides of this equation on
the left by $b$ and on the right by $b^{-1}$, we obtain $H\subseteq
bHb^{-1}$.  Hence, since $bHb^{-1}\subseteq H$ and $H\subseteq
bHb^{-1}$, those two sets are equal. Thus, Statement 2 is proven.\end{proof}

We now consider some examples and nonexamples of normal subgroups of
groups.

\begin{example}{}\
\begin{enumerate}
\item As previously mentioned, the trivial and improper subgroups of
any group $G$ are normal in $G$.

\item As previously mentioned, if group $G$ is abelian then each of
its subgroups is normal in $G$.

\item Suppose $H\leq G$ has $(G:H)=2$.  Then $H \unlhd G$. \red{The proof of this is left as an exercise for the reader.}

\item Example \ref{s3.ex} shows that subgroup $H=\<(12)\>$ isn't normal in
$S_3$ (for example, $(13)H\neq H(13)$.  But $\<(123)\>$ must be
normal in $S_3$ since $(S_3:\<(123)\>)=6/3=2.$

\item $\<r\>$ is normal in $D_n$ since $(D_n:\<r\>)=2$.

\item $\<f\>$ isn't normal in $D_4$: for instance,
$$r\<f\>r^{-1}=\{e,rfr^3\}=\{e, fr^3r^3\}=\{e,fr^2\}\not\subseteq
\<f\>. $$
\end{enumerate}
\end{example}

 We consider two other very significant examples.


\begin{df}{Definition, notation, and terminology} Let $G$
be a group. We let $$Z(G)=\{z\in G\,:\, az=za \mbox{ for all }a\in
G\}.$$ $Z(G)$ is called the \textit{center} of $G$.\footnote{The center of a group was originally introduced in the Exercises of Chapter \ref{gps}. The Z stands for ``zentrum," the German word for ``center."}\end{df}

\begin{thm}\label{} Let $G$ be a group.  Then $Z(G)$ is a subgroup of $G$.
\end{thm}

\begin{proof}

 Clearly,
$e\in Z(G)$.  Next, let $z,w\in Z(G)$. Then for all $a \in G$,
$$a(zw)=(az)w=(za)w=z(aw)=z(wa)=(zw)a,$$ so $zw\in Z(G)$.  Finally, $az=za$ so, multiplying both
sides on the left and right by $z^{-1}$, we have $z^{-1}a=az^{-1}$;
thus, $z^{-1}\in Z(G)$.  Hence, $Z(G)\leq G$. \end{proof}

\begin{thm}\label{znorm}Let $G$ be a group and let $H$ be a subgroup of $G$ with $H\subseteq Z(G)$. Then $H\unlhd G$. In particular, $Z(G)$ is itself a normal
subgroup of $G$.\end{thm}

\begin{proof} Let $a\in G$.
Then since every element of $Z(G)$ commutes with every element of
$G$, every element of $H$ commutes with every element of $G$; so we must have $aH=Ha$.  Thus,
$H\unlhd G$. \end{proof}

 The next definition is profoundly important for us.

\begin{df}{Definition and notation} Let $G$ and $G'$ be groups and let $\phi$ be a
homomorphism from $G$ to $G'$. Letting $e'$ be the identity
element of $G'$, we define the \textit{kernel of $\phi$}, $\Ker
\phi$, by
$$\Ker \phi = \{k\in G\,:\,\phi(k)=e'\}.$$\end{df}

 \begin{example}{} Let $G$ and $G'$ be groups and let $\phi$ be a
homomorphism from $G$ to $G'$. Then $\Ker \phi$ is a normal subgroup
of $G$.

\begin{proof} Let $K=\Ker \phi$.  We first show that $K$ is a subgroup of
$G$. Clearly, the identity element of $G$ is in $K$, so $K\neq
\emptyset$. Next, let $k,m\in K$.  Then, letting $e'$ denote the
identity element of $G'$, we have
$$\phi(km^{-1})=\phi(k)\phi(m)^{-1}=e'(e')^{-1}=e',$$ so $km^{-1}\in
K$.  Thus, by the Two-Step Subgroup Test, we have that $K$ is a
subgroup of $G$.

Next, let $k\in K$ and let $a\in G$. Then
$$\phi(aka^{-1})=\phi(a)\phi(k)\phi(a)^{-1}=\phi(a)e'\phi(a)^{-1}=\phi(a)\phi(a)^{-1}=e'.$$
So $aka^{-1}\in K$. Thus, $K \unlhd G$.\end{proof} \end{example}

One slick way, therefore, of showing that a particular set is a
normal subgroup of a group $G$ is by showing it's the kernel of a
homomorphism from $G$ to another group.

\begin{example}{slnormgl} Let $n\in \Z^+$. Here is a rather elegant proof of the fact that
$SL(n,\fatr)$ is a normal subgroup of $GL(n,\fatr)$:  Define $\phi:
GL(n,\fatr) \to \fatr^*$ by $\phi(A)=\det A$.  Clearly, $\phi$ is a
homomorphism, and since the identity element of $\fatr^*$ is 1,
$$\Ker \phi=\{A\in GL(n,\fatr)\,:\,\det A= 1\}=SL(n,\fatr).$$ So
$SL(n,\fatr)\unlhd GL(n,\fatr)$.
\end{example}


\section{Factor groups}

 We now return to the notion
of equipping $G/H$, when $H\unlhd G$, with a group structure.
We have already proven that left coset multiplication on $G/H$
is well defined when $H\unlhd G$; it turns out that given this,
it is very easy to prove that $G/H$ under this operation is a
group.

 Before we prove this, we  introduce a change in our notation: We
initiate the convention of frequently using $N$, rather than $H$, to
denote a subgroup of a group $G$ when we know that that subgroup is
\underline{normal} in $G$.

\begin{thm}\label{} Let $G$ be a group with identity element $e$, and let $N\unlhd G$.  Then $G/N$ is a
group under left coset multiplication (that is, under the operation
$(aN)(bN)=abN$ for all $a,b\in G$), and $|G/N|=(G:N)$ (in
particular, if $|G|<\infty$, then $|G/N|=|G|/|N|$).
\end{thm}

\begin{proof} We already know that since $N\unlhd G$, left coset
multiplication on $G/N$ is well defined; further, it is clear
that $G/N$ is closed under this operation.

Associativity of this operation on $G/N$ follows from the
associativity of $G$'s operation: indeed, if $aN$, $bN$, $cN \in
G/N$, then
$$((aN)(bN))(cN)=(abN)(cN)=(ab)cN=a(bc)N=(aN)(bcN)=(aN)((bN)(cN)).$$

Next, $N=eN\in G/N$ acts as an identity element since for all $a\in
G$,

\begin{center}$(aN)(N)=aeN=aN$ and $N(aN)=eaN=aN$.\end{center}

Finally, let $aN\in G/N$. Then $a^{-1}N\in G/N$ with
$(aN)(a^{-1}N)=aa^{-1}N=N$ and $(a^{-1}N)(aN)=a^{-1}aN=N$.

So $G/N$ is a group under left coset multiplication. Since $(G:N)$
is, by definition, the cardinality of $G/N$, we're done.\end{proof}

\begin{df}{Definition} When $G$ is a group and $N\unlhd G$, the above group
($G/N$ under left coset multiplication) is called a \textit{factor
group} or \textit{quotient group}.\end{df}

\begin{example}{} Let $G=\Z$ and $N=3\Z$.  Then $N$ is normal in $G$, since $G$ is abelian, so the set
$G/N=\{N,1+N,2+N\}$ is a group under left coset multiplication.
 Noting that $N=0+N$, it is straightforward to see that
 $G/N$ (that is, $\Z/3\Z$) has the following group table:

\bigskip
\begin{center}
\renewcommand{\arraystretch}{1.3}
$\begin{array}{c||c|c|c} +&0+N&1+N&2+N\\ \hline\hline
0+N&0+N&1+N&2+N\\ \hline 1+N&1+N&2+N&0+N\\ \hline
2+N&2+N&0+N&1+N\\
\end{array}$
\end{center}
 Clearly, if we ignore all the $+N's$ after the $0$'s, $1$'s, and
$2'$ in the above table, and consider $+$ to be addition mod 3,
rather than left coset addition in $\Z/3\Z$, we obtain the group
table for $\Z_3$:
\end{example}

\bigskip\begin{center}
\renewcommand{\arraystretch}{1.3}
$\begin{array}{c||c|c|c} +&0&1&2\\ \hline\hline 0&0&1&2\\ \hline
1&1&2&0\\ \hline 2&2&0&1\\
\end{array}$\end{center}

 Thus, we see that $\Z/3\Z$ is isomorphic to $\Z_3$.
 This is not a coincidence!  In fact, we have the following:

\begin{thm}\label{} Let $n,d \in \Z^+$ with $d$ dividing $n$. Then we have:

\begin{enumerate}
\item $n\Z\unlhd d\Z$ and $\<d\>\unlhd \Z_n$;

\item $d\Z/n\Z\simeq \Z_{n/d}$ (special case: $\Z/n\Z \simeq \Z_n$);
and

\item  $Z_n/\<d\> \simeq \Z_d$.
\end{enumerate}
\end{thm}

\begin{proof}\

\begin{enumerate}
\item Since $d$ is a positive divisor of $n$, $n\Z$ and
$\<d\>$ are clearly subgroups of, respectively, $d\Z$ and $\Z_n$.
Moreover, since $d\Z$ and $\Z_n$ are abelian, all of their subgroups
are normal.

\item This follows from the facts that $d\Z/n\Z=\<d+n\Z\>$, hence
is cyclic, and that $|d\Z/n\Z|=(d\Z:n\Z)=n/d$ (see Example
\ref{indices.ex}). (Unpacking the statement that
$d\Z/n\Z=\<d+n\Z\>$: Notice that $d\Z=\<d\>$, so every element of
$d\Z$ is of the form $kd$ for some integer $k$. Thus, every element
of $d\Z/n\Z$ is of the form $kd+n\Z$ for some integer $k$. But for
each $k\in \Z$, using the definition of left coset multiplication we
have that $kd+n\Z=k(d+n\Z)$. So $d\Z/n\Z=\<d+n\Z\>$.)

\item Similarly, since $\Z_n=\<1\>$, $\Z_n/\<d\>=\<1+\<d\>\>$, so
is cyclic. Since $(\Z_n:\<d\>)=d$ (again, see Example
\ref{indices.ex}), $\Z_n/\<d\>$ is isomorphic to $\Z_d$, as desired. \qedhere
\end{enumerate}
\end{proof}


\begin{example}{} Let $N=\<(123)\> \leq S_3$.  Since $(S_3:N)=2$,
$N$ must be normal in $S_3$, so $S_3/N$ is a group under left coset
multiplication.  Since $|S_3/N|=2$, $S_3/N$ must be isomomorphic to
$\Z_2$. \end{example}

In the above examples, we were able to identify $G/N$ up to
isomorphism relatively easily because we could determine that $G/N$
was cyclic.  In general, however, it can be quite difficult to
identify the group structure of a factor group.  We explore some
powerful tools we can use in this identification in the next
section, but first we note a couple properties of $G$ that are
``inherited" by any of its factor groups.

\begin{thm}\label{} Let $G$ be a group and $N\unlhd G$. Then:
\begin{enumerate}\item If $G$ is finite, then $G/N$ is finite.

\item If $G$ is abelian, then $G/N$ is abelian; and

\item If $G$ is cyclic, then $G/N$ is cyclic.
\end{enumerate}
\end{thm}

\begin{proof}\

\begin{enumerate}
\item
Clearly this holds, since in this case
$|G/N|=|G|/|N|$.

\item \red{The proof of this statement is left as an exercise for the reader.}

\item Let $G$ be cyclic.  Then there exists $a\in G$ with $G=\<a\>$.  We claim $G/N=\<aN\>$. Indeed:
$$\<aN\>=\{(aN)^i\,:\,i\in \Z\}=\{a^iN\,:\,i\in \Z\}.$$ But every element of $G$ is an integer power of $a$, so this set equals $\{xN\,:x\in G\}$, that is, it equals $G/N$. \qedhere

\end{enumerate} \end{proof}

 \warn{However, $G$ can be nonabelian [noncyclic, nonfinite] and
yet have a normal subgroup $N$ such that $G/N$ is abelian [cyclic,
finite]. (See the examples below.) Intuitively, the idea is that
modding out a group by a normal subgroup can \underline{introduce}
abelianness or cyclicity, or finiteness, but not \underline{remove}
one of those characteristics.}

\begin{example}{} $S_3$ is nonabelian (and therefore of course noncyclic), but we
saw above that $N=\<(123)\>$ is a normal subgroup of $S_3$ with
$S_3/N \simeq \Z_2$, a cyclic (and therefore of course abelian)
group. \end{example}

\begin{example}{} $\Z$ is an infinite group, but it has finite factor group
$\Z/2\Z$. \end{example}

What do the (normal) subgroups of a factor group $G/N$ look like?
Well, they come from the (normal) subgroups of $G$!  We have the
following theorem, whose proof is tedious but straightforward.

\begin{thm}\label{}{\textbf{(Correspondence Theorem)}} Let $G$ be a
group, and let $K\unlhd G$.  Then the subgroups of $G/K$ are exactly
those of the form $H/K$, where $H\leq G$ with $K\subseteq H$.
Moreover, the normal subgroups of $G/K$ are exactly those of the
form $N/K$, where $N\unlhd G$ with $K\subseteq N$.\end{thm}


\begin{example}{} Let $N=18\Z$ in $\Z$. Since the subgroups of $\Z$ containing $N$
are the sets $d\Z$ where $d$ is a positive divisor of $18$, the
subgroups of $\Z/N$ are the sets $d\Z/N$, where, again, $d$ is a
positive divisor of $18$. \end{example}

 We end this chapter by noting that given any group $G$ and
factor group $G/N$ of $G$, there is a homomorphism from $G$ to $G/N$
that is onto. Before we define this homomorphism, we provide some
more terminology.

\begin{df}{Definitions} Let $\phi: G\to G'$ be a homomorphism of groups. Then
$\phi$ is can be called an \textit{epimorphism} if $\phi$ is onto,
and a \textit{monomorphism} if $\phi$ is one-to-one. (Of course,
we already know that an epimorphism that is also a monomorphism
is called an ``isomorphism.")\end{df}

 We now present the following theorem, whose straightforward
proof is left to the reader.

 \begin{thm}\label{}  Let $G$ be a group and let $N\unlhd G$. Then the
map $\Psi: G\to G/N$ defined by $\Psi(g)=gN$ is an epimorphism. \end{thm}

\begin{df}{Definition} We call this map $\Psi$ the \textit{canonical epimorphism
from $G$ to $G/N$}.\end{df}

 Notice that given $N\unlhd G$, the kernel of the canonical
epimorphism from $G$ to $G/N$ is $N$.  Thus, putting this fact
together with the fact that every kernel of a homomorphism is a
normal subgroup of the homomorphism's domain, we have the following:

\begin{thm}\label{} Let $G$ be a group and $N$ a subset of $G$.  Then $N$
is a normal subgroup of $G$ if and only if $N$ is the kernel of a
homomorphism from $G$ to some group $G'$. \end{thm}

\pagebreak

\section{Exercises}

\begin{exercise} Show that if $H$ is a subgroup of index 2 in a group
    $G$, then $H$ is normal in $G$.
\end{exercise}

\begin{solution}[print=true]
Since $(G:H)=2$, there are exactly two left
    cosets of $H$ in $G$.  One of these left cosets is
    $H$; since the left cosets of $H$ partition $G$,
    the remaining coset must be $G-H$.  So if $a\in G$,
    $aH=H$ if $a\in H$, and $aH=G-H$ if $a\notin H$.
    Similarly, we can show that $Ha=H$ if $a\in H$, and
    $Ha=G-H$ if $a\notin H$. Thus, $aH=Ha$ for every
    $a\in G$, so $H$ is normal in $G$.
\end{solution}

\begin{exercise}
Let $H=\{A\in GL(n,\fatr)\,:\,\det A=\pm 1\}$.  Prove that $H\unlhd GL(n,\fatr)$.

\end{exercise}

\begin{solution}[print=true]
Consider the function $\phi \,:\,GL(n, \fatr)\to \fatr^*$ defined by
$\phi(A)=|\det A|$. The map $\phi$ is a
homomorphism, since if $A,B\in GL(n,\fatr)$, then
$$\phi(AB)=|\det(AB)|=|\det A||\det B|=\phi(A)\phi(B).$$  Further,
\begin{align*}\Ker \phi&=\{A\in GL(n,\fatr)\,:\,
\phi(A)=1\}\\&=\{A\in GL(n,\fatr)\,:\, |\det A|=1\}\\&=\{A\in
GL(n,\fatr)\,:\, \det A=\pm 1\}\\&=H.\end{align*} Since $H$ is
the kernel of a homomorphism from $GL(n,\fatr)$ to another group, it must be a
normal subgroup of $G$.
\end{solution}



\begin{exercise} Let $G=\Z_6\times \Z_{14}$ and let $H=\<2\>\times \<4\>\leq G$.  Since $G$ is abelian, $H\unlhd G$.  Find the order of the factor group $|G/H|$.
\end{exercise}

\begin{solution}[print=true]
Note that $|G|=6(14)=84$, and since $\<2\>=\{2,4,0\}$ and $\<4\>=\{4,8,12,2,6,10,0\}$, $|H|=3(7)=21$. So $|G/H|=|G|/|H|=84/21=4$.
\end{solution}



\begin{exercise}
Consider the subgroup $\<18\>$ of $\Z$.

\begin{enumerate}
\item Explain how you know that $\Z/\<18\>$ is a group under left coset multiplication.
\item Find:

\begin{enumerate}
\item $|\Z/\<18\>|$.
\item $|4+\<18\>|$.
\item $o(4+\<18\>)$ in $\Z/\<18\>$.
\end{enumerate}
\end{enumerate}

\end{exercise}


\begin{solution}[print=true]
Note that in $\Z$, $\<18\>=18\Z$, so we may denote $\<18\>$ by $18\Z$ throughout this problem.

\begin{enumerate}
\item $\Z/18\Z$ is a group since $18\Z$ is a subgroup of $\Z$, and $\Z$ is abelian, so every subgroup of $\Z$ is normal in $\Z$.
\item

\begin{enumerate}
\item $|\Z/18\Z|=(\Z:18\Z)=18$.
\item $|4+18\Z|=\infty$ (since $4+18\Z=\{\ldots, -14, 4, 22,\ldots\}$).
\item The subgroup generated by $4+18\Z$ in $\Z/18\Z$ is
$$\{4+18\Z, 8+18\Z, 12+18\Z, 16+18\Z, 20+18\Z, 24+18\Z, 28+18\Z, 32+18\Z, 18\Z\},$$ so $o(4+18\Z)=9$.
\end{enumerate}
\end{enumerate}

\end{solution}

\begin{comment}
\begin{exercise}
Find the order of some $(\Z_n \times \Z_m )/\<(a,b)\>$.
\end{exercise}
\end{comment}

\begin{exercise} Let $G=4\Z \times \M_2(\fatr)$, $H=\<2\>\times \<I_2\> \leq G$, and $K=\<(23)\>\leq S_3$.
Explain how you know that the set $G/H$ is a group under left coset multiplication, while the set $S_3/K$ is not.
\end{exercise}

\begin{solution}[print=true]

Since $G$ is abelian, $H$ is normal in $G$, so $G/H$ is a group under left coset multiplication.  Since $K$ isn't normal in $S_3$ (for instance, $(12)K=\{(23),(13)\}$ while $K(12)=\{(12),(13)\}$), $S_3/K$ isn't a group under left coset multiplication.
\end{solution}

\begin{exercise} Let $H=\<r^2\>\leq D_4$.
\begin{enumerate}
\item Prove that $H\unlhd D_4$. (Hint: Refer back to Section \ref{dihedralgps}.)
\item It follows that $D_4/H$ is a group under left coset multiplication.
\begin{enumerate}
\item  The distinct left cosets of $H$ in $D_4$ are $H$, $rH$, $fH$, and $frH$. Explicitly list the elements of each of these cosets of $H$.
\item Complete the group table for $D_4/H$, denoting each coset by $H$, $rH$, $fH$, or $frH$, as appropriate.
\item Use the group table for $D_4/H$ to identify a familiar group to which $D_4/H$ is isomorphic.
\end{enumerate}
\end{enumerate}

\end{exercise}


\begin{solution}[print=true]

\begin{enumerate}
\item Note that $H=\{e,r^2\}$. By the last statement in Theorem \ref{diords}, $r^2$ commutes with every element of $D_4$; moreover, $e$ clearly commutes with every element of $D_4$. Hence, $H\subseteq Z(D_4)$, and so by Theorem \ref{znorm}, $H\unlhd D_4$.

\item \begin{enumerate}
\item The left cosets of $H$ in $D_4$ are


$$\begin{array}{rclcl}
H&=&\{e,r^2\}&=&r^2H,\\
rH&=&\{r,r^3\}&=&r^3H,\\
fH&=&\{f, fr^2\}&=&fr^2H,\\
\mbox{and \,}frH&=&\{fr,fr^3\}&=&fr^3H.
\end{array}$$

\item  \

\renewcommand{\arraystretch}{1.3}
$$\begin{array}{r||r|r|r|r}
\cdot &H&rH&fH&frH\\ \hline\hline H &H&rH&fH&frH\\ \hline
rH&rH&H&frH&fH\\ \hline fH&fH&frH&H&rH\\ \hline frH&frH&fH&rH&H

\end{array}$$

\item Since $|D_4/H|=4$ and every element of $D_4/H$ is its own inverse, $D_4/H\simeq Z_2^2$.
\end{enumerate}
\end{enumerate}
\end{solution}


\begin{exercise}
Let $G$ be an abelian group and let $N\unlhd G$. Prove that $G/N$ is abelian.
\end{exercise}

\begin{solution}[print=true]
Let $aN, bN\in G/N$.  Then $(aN)(bN)=abN=baN=(bN)(aN)$ (the first and last equalities follow from the definition of left coset multiplication, and the second equality holds because $G$ is abelian).  So $G/N$ is abelian.
\end{solution}

\begin{exercise} Let $N$ be a normal subgroup of group $G$, with finite index $m$.  Let $a\in G$. Prove that $a^m\in N$.
\end{exercise}

\begin{solution}[print=true]
Since $m=(G:N)$, $|G/N|=m$.  The element
$aN\in G/N$ has an order, call it $d$, that must divide $|G/N|=m$.
So $m=dk$ for some $k\in \Z$.  Then
$(aN)^m=(aN)^{dk}=((aN)^d)^k=N^k=N$. Since $(aN)^m=a^mN$, we thus
have $a^mN=N$, implying that $a^m\in N$, as desired.\end{solution}